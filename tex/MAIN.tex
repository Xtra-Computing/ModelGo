%%
%% This is file `sample-sigconf.tex',
%% generated with the docstrip utility.
%%
%% The original source files were:
%%
%% samples.dtx  (with options: `sigconf')
%% 
%% IMPORTANT NOTICE:
%% 
%% For the copyright see the source file.
%% 
%% Any modified versions of this file must be renamed
%% with new filenames distinct from sample-sigconf.tex.
%% 
%% For distribution of the original source see the terms
%% for copying and modification in the file samples.dtx.
%% 
%% This generated file may be distributed as long as the
%% original source files, as listed above, are part of the
%% same distribution. (The sources need not necessarily be
%% in the same archive or directory.)
%%
%%
%% Commands for TeXCount
%TC:macro \cite [option:text,text]
%TC:macro \citep [option:text,text]
%TC:macro \citet [option:text,text]
%TC:envir table 0 1
%TC:envir table* 0 1
%TC:envir tabular [ignore] word
%TC:envir displaymath 0 word
%TC:envir math 0 word
%TC:envir comment 0 0
%%
%%
%% The first command in your LaTeX source must be the \documentclass
%% command.
%%
%% For submission and review of your manuscript please change the
%% command to \documentclass[manuscript, screen, review]{acmart}.
%%
%% When submitting camera ready or to TAPS, please change the command
%% to \documentclass[sigconf]{acmart} or whichever template is required
%% for your publication.
%%
%%
\documentclass[sigconf]{acmart}
%\documentclass[sigconf, nonacm]{acmart}


\copyrightyear{2024}
\acmYear{2024}
\setcopyright{rightsretained}
\acmConference[WWW '24]{Proceedings of the ACM Web Conference 2024}{May 13--17, 2024}{Singapore, Singapore}
\acmBooktitle{Proceedings of the ACM Web Conference 2024 (WWW '24), May 13--17, 2024, Singapore, Singapore}\acmDOI{10.1145/3589334.3645520}
\acmISBN{979-8-4007-0171-9/24/05}

\settopmatter{printacmref=true}

%%
%% Submission ID.
%% Use this when submitting an article to a sponsored event. You'll
%% receive a unique submission ID from the organizers
%% of the event, and this ID should be used as the parameter to this command.
%%\acmSubmissionID{123-A56-BU3}

%%
%% For managing citations, it is recommended to use bibliography
%% files in BibTeX format.
%%
%% You can then either use BibTeX with the ACM-Reference-Format style,
%% or BibLaTeX with the acmnumeric or acmauthoryear sytles, that include
%% support for advanced citation of software artefact from the
%% biblatex-software package, also separately available on CTAN.
%%
%% Look at the sample-*-biblatex.tex files for templates showcasing
%% the biblatex styles.
%%

%%
%% The majority of ACM publications use numbered citations and
%% references.  The command \citestyle{authoryear} switches to the
%% "author year" style.
%%
%% If you are preparing content for an event
%% sponsored by ACM SIGGRAPH, you must use the "author year" style of
%% citations and references.
%% Uncommenting
%% the next command will enable that style.
%%\citestyle{acmauthoryear}

\usepackage{comment}
\usepackage{multirow}
\usepackage{amsmath,pifont}
\usepackage{booktabs}
\usepackage{rotating}
\usepackage{array}
\usepackage[para]{footmisc}
\usepackage{tablefootnote}
\usepackage{xcolor}
\usepackage{colortbl}
\usepackage{longtable}
\usepackage{circledsteps}
\usepackage{romannum}
\usepackage{longtable}
\usepackage{tcolorbox}
\usepackage{enumitem} % no indent [leftmargin=*]
\usepackage{balance}

\definecolor{Permissive}{HTML}{18AFCC}
\definecolor{Public}{HTML}{5F9099}
\definecolor{Copyleft}{HTML}{CC1865}

\hyphenation{OpenAI Model-Go} 
%%
%% end of the preamble, start of the body of the document source.
\begin{document}

%%
%% The "title" command has an optional parameter,
%% allowing the author to define a "short title" to be used in page headers.
\title{ModelGo: A Practical Tool for Machine Learning License Analysis}

%%
%% The "author" command and its associated commands are used to define
%% the authors and their affiliations.
%% Of note is the shared affiliation of the first two authors, and the
%% "authornote" and "authornotemark" commands
%% used to denote shared contribution to the research.

\author{Moming Duan}
\affiliation{%
  \institution{National University of Singapore}
  %\city{Singapore}
  \country{Singapore}}
\email{moming@nus.edu.sg}


\author{Qinbin Li}
\affiliation{%
  \institution{UC Berkeley}
  \city{Berkeley}
  \country{USA}}
\email{qinbin@berkeley.edu}

\author{Bingsheng He}
\affiliation{%
  \institution{National University of Singapore}
  %\city{Singapore}
  \country{Singapore}}
\email{hebs@comp.nus.edu.sg}

%%
%% By default, the full list of authors will be used in the page
%% headers. Often, this list is too long, and will overlap
%% other information printed in the page headers. This command allows
%% the author to define a more concise list
%% of authors' names for this purpose.
\renewcommand{\shortauthors}{Moming Duan, Qinbin Li, \& Bingsheng He}

%%
%% The abstract is a short summary of the work to be presented in the
%% article.
\begin{abstract}
Productionizing machine learning projects is inherently complex, involving a multitude of interconnected components that are assembled like LEGO blocks and evolve throughout development lifecycle.
These components encompass software, databases, and models, each subject to various licenses governing their reuse and redistribution.
However, existing license analysis approaches for Open Source Software (OSS) are not well-suited for this context.
For instance, some projects are licensed without explicitly granting sublicensing rights, or the granted rights can be revoked, potentially exposing their derivatives to legal risks.
Indeed, the analysis of licenses in machine learning projects grows significantly more intricate as it involves interactions among diverse types of licenses and licensed materials.
To the best of our knowledge, no prior research has delved into the exploration of license conflicts within this domain.
In this paper, we introduce ModelGo, a practical tool for auditing potential legal risks in machine learning projects to enhance compliance and fairness.
With ModelGo, we present license assessment reports based on five use cases with diverse model-reusing scenarios, rendered by real-world machine learning components.
Finally, we summarize the reasons behind license conflicts and provide guidelines for minimizing them.
Our code is publicly available at \url{https://github.com/Xtra-Computing/ModelGo}.
\end{abstract}

%bundling the LGPL-LR corpus dataset with another corpus dataset licensed under CC-BY-SA 4.0 can result in a copyleft proliferation conflict. %as stipulated by the derivative work terms of LGPL-LR.
%In contrast, bundling this corpus with a model licensed under copyleft GPL 3.0 does not lead to such conflict, as the derivative work is no longer considered a linguistic resource.

%%
%% The code below is generated by the tool at http://dl.acm.org/ccs.cfm.
%% Please copy and paste the code instead of the example below.
%%
\begin{CCSXML}
  <ccs2012>
     <concept>
         <concept_id>10011007.10011074.10011134.10003559</concept_id>
         <concept_desc>Software and its engineering~Open source model</concept_desc>
         <concept_significance>300</concept_significance>
         </concept>
     <concept>
         <concept_id>10003456.10003457.10003580.10003585</concept_id>
         <concept_desc>Social and professional topics~Testing, certification and licensing</concept_desc>
         <concept_significance>500</concept_significance>
         </concept>
   </ccs2012>
\end{CCSXML}

\ccsdesc[500]{Social and professional topics~Testing, certification and licensing}
\ccsdesc[300]{Software and its engineering~Open source model}

%%
%% Keywords. The author(s) should pick words that accurately describe
%% the work being presented. Separate the keywords with commas.
\keywords{License analysis, AI licensing, model mining}

%\received{20 February 2007}
%\received[revised]{12 March 2009}
%\received[accepted]{5 June 2009}

%%
%% This command processes the author and affiliation and title
%% information and builds the first part of the formatted document.
\maketitle

\section{Introduction}
% 人工智能基础设施的快速发展和产品化大大加速了机器学习组件数量的增长,模型的复用,如finetune和moe变得常见
Over the past decade, the advancement and productization of AI infrastructures have significantly accelerated the proliferation of machine learning (ML) components~\cite{jiang2023empirical}, including AI models~\cite{rombach2022high, touvron2023llama}, software~\cite{wolf2020transformers, he2022fastermoe}, and big datasets~\cite{gao2020the, schuhmann2022laion}.
Concurrently, the reuse of these components has gained popularity, motivated by concerns about their significant demands on financial and energy resources~\cite{strubell2019energy}, as well as the widespread recognition of the value advocated by the open-source movement~\cite{rosen2005open}.
Unlike code reuse in the OSS field, the reuse of AI models follow a distinct schema.
A frequently employed approach for AI models reuse is fine-tuning Pre-Trained Models (PTMs)~\cite{han2021pre, touvron2023llama}, where PTMs are adapted on a domain-specific dataset, leveraging their robust generalization capabilities. 

% 但是由于xxx原因导致software,data,model都有不同的license,存在潜在的冲突和法律风险。
From a legal perspective, model reuse is generally uncontroversial when its developers or affiliated companies own the copyright for all components.
However, data and models often have separate copyright holders in nowadays ML projects~\cite{rajbahadur2021can, radford2019language, scao2022bloom, zeng2023glm}.
For instance, GPT-2~\cite{radford2019language}, developed by OpenAI, was trained on 45 million web pages containing content from third-party platforms like WordPress, GitHub, and IMDb, none of which is owned by OpenAI.
These crowdsourced content typically provides limited usage and distribution rights to users through pre-agreed licenses (e.g., Creative Commons Licenses\footnote{https://creativecommons.org/licenses/}), which may restrict certain reuse methods like remixing, reproducing, and translating. 
To prevent legal risk, it is essential to ensure that the final ML projects remain compaliant with all license conditions associated with the reused components~\cite{cui2023empirical, mathur2012empirical, kapitsaki2017automating}.

However, compared to assessing licensing compliance for OSS, ensuring license compliance in ML projects poses several unique challenges. 
First, a ML project is not only a combination of software like an OSS project but also composed of datasets and models~\cite{han2021pre}, which may be under different types of licenses (e.g., Free Content Licenses and AI model licenses~\cite{contractor2022behavioral}).
Second, ML components often follow more complicated coupling paradigms and nested workflows. For instance, Openjourney\footnote{https://openjourney.art/} is an image generation model derived from StableDiffusion~\cite{rombach2022high}, and fine-tuned on images generated by another commercial product, Midjourney\footnote{https://www.midjourney.com/}.
This demonstrates that knowledge can be transferred between models without explicit code integration~\cite{you2021workshop}.
Another challenge is improper and ambiguity licensing in ML projects.
For example, GPT-2 and BERT~\cite{devlin2019bert} are regarded as part of software and then licensed as OSS (e.g., MIT and Apache-2.0).
However, ML projects like StableDiffusion and Llama2~\cite{touvron2023llama} tend to apply responsible AI restriction terms for both model and code, using AI model licenses such as OpenRAIL-M~\cite{contractor2022behavioral} and Llama2 Community License\footnote{https://huggingface.co/meta-llama/Llama-2-7b}.
Additionally, to circumvent the limitations of standard OSS licenses, some licensors adopt non-commercial content licenses or custom licenses to protect the Intellectual Property (IP) of their models by prohibiting commercial use~\cite{huang2022layoutlmv3}, fine-tuning~\cite{dreamlike2023}, and reverse engineering~\cite{goyal2022vision}.
Such ambiguity and the diverse licensing practices within ML projects  increase significant legal uncertainty in license compliance analysis.
As a result, traditional OSS license analysis approaches~\cite{ombredanne2020free, mathur2012empirical} only consider inclusion and linking relationships among software and lack support for AI model licenses, making them unsuitable for ML project license analysis.

In this paper, we introduce Modelgo, a tool designed to analyze potential license conflicts, improper license choices, use restrictions and obligations in ML projects that involve nested component reuse procedures.
To demonstrate the usefulness of Modelgo, we present 5 use cases constructed using 15 datasets and 11 models from real-world, whose license types cover OSS, free content, and AI model.
Our findings show that there exist potential legal risks when reusing  components under copyleft or non-commercial licenses, and point out the need for attention to AI model licenses.
The main contributions of our paper are:
\begin{itemize}
    \item We raise the challenge of license analysis for ML projects and propose Modelgo to assessing it. To the best of our knowledge, our work is the first attempt to deal with this challenge in the ML context.
    \item As part of our work, we introduce a new taxonomy based on the forms of reused components to identify the corresponding conditions for various ML reuse mechanisms. This method helps mitigate ambiguity in cases of mismatch between applied license type and actual component type, allowing Modelgo to analyze components under various license types, including OSS, free content and AI models.
    \item  We provide legal compliance assessment reports based on 5 use cases to showcase the effectiveness of our approach. 
    Through our use cases, we offer valuable insights and experiences in achieving legal compliance in ML projects. 
    Additionally, we also provide license choosing recommendations to minize the risk of non-compliance.
\end{itemize}

The rest of the paper is organized as follows. (TBD)
%We first propose using a new taxonomy based on the forms of reused components to identify the corresponding conditions for different reuse mechanisms.
%Based on Modelgo employs a tree structure to track the dependencies of components in ML pipelines.

% A Large-scale Dataset of (Open Source) License Text Variant, 有很多license为other
% An Empirical Study of License Conflict in Free and Open Source Software
% Open Source License Inconsistencies on GitHub
% An empirical study of license violations in open source projects
% Do Software Developers Understand Open Source Licenses?
% Analyzing Open Source License Compatibility Issues with Carneades

\begin{comment}
There is no consensus on whether the use of copyright works as input to train an AI system is an exercise of an exclusive right.
There remains significant legal uncertainty about whether copyright applies to AI training, which means it may not always be clear whether a CC license applies.
The larger model was trained on 256 cloud TPU v3 cores. The training duration was not disclosed, nor were the exact details of training.

Open source software license compliance~\cite{ombredanne2020free}

The open source definition~\cite{perens1999open}

AFL~\cite{rosen2005open}

Wudao2.0 1.75T MoE
[FASTMOE: A FAST MIXTURE-OF-EXPERT TRAINING SYSTEM]
[GLM-130B: AN OPEN BILINGUAL PRE-TRAINED MODEL]

Objectives and challenges associated with analyzing dataset license compliance?
Getty Images (US), Inc. v. Stability AI, Inc. (1:23-cv-00135)
Andersen et al v. Stability AI Ltd. et al (3:23-cv-00201)
We are not aware of any copyright restrictions of the material

C4, Pile Common Crawl
crowdsourced

COCO (CC-BY 4.0), CIFAR10 -> Flickr
Unsplash License \textit{Custom}: Compiling photos from Unsplash to replicate a similar or competing service. https://unsplash.com/license
Pixabay License: Data mining, extraction, scraping and the use of programs or robots for automatic data collection and/or extraction of digital data on the Services and/or the content available therein is strictly prohibited for all purposes, including without limitation for machine learning purposes.

Google Street View (SVHN) https://about.google/brand-resource-center/products-and-services/geo-guidelines/

%实际上,license analysis应该可分为四层,最底层是license 是否 compliance (或者叫做dataset provenance extraction, License identification),例如数据集的license是否可以覆盖每个sample,还有就是distribution的版本的license和official source的license不一致,第二层是license之间是否conflict,第三层是应用所需的rights是否具备,最顶层是应用场景是否符合法律法规(regulation)


Software reuse is very simple from the legal point of view, if a company or an
individual reuses software for which it has copyrights. However, things change dramatically
if one wants to reuse software made by others, since software is protected
by copyright and possibly by patents. Without explicit permission, no person other
than the copyright holder is allowed to copy, distribute, or make derivative works
from the original work.
\end{comment}

\begin{table*}[]
    \caption{Summary of machine learning projects in Huggingface. }
    \footnotesize
    \label{tab:MLP}
    \begin{tabular}{|p{2.1cm}|p{1.6cm}|p{2cm}|p{2.75cm}|p{3cm}|p{1.7cm}|p{2cm}|}
        \hline
        \textbf{ML Project} & \textbf{Task} & \textbf{Data License} & \textbf{Software License} & \textbf{Model License} & \textbf{Dataset} & \textbf{Risk Resource} \\ \hline
        
        Stable Diffusion v1-5 & Text to Image & CC-BY-4.0 & CreativeML-OpenRAIL-M & CreativeML-OpenRAIL-M & LAION-5B & Common Crawl \\ \hline
        
        BLOOM & Text Generation & \textit{Mixture} & \textit{Unknown} & BigScience-BLOOM-RAIL-1.0 & \textit{Crowdsourced} & Common Crawl, \newline Wikipedia, etc. \\ \hline

        OrangeMixs & Text to Image & \textit{Mixture} & \textit{Unknown} & CreativeML-OpenRAIL-M & \textit{Crowdsourced} & Danbooru \\ \hline

        ControlNet & Text to Image &  \textit{Unknown} & Apache-2.0 & OpenRAIL & \textit{Unknown} & n/a \\ \hline

        Openjourney & Text to Image &  CC-BY-NC-4.0 & \textit{Unknown} & CreativeML-OpenRAIL-M & Midjourney Gen & Midjourney Gen \\ \hline

        ChatGLM-6B & Text Generation &  \textit{Mixture} & Apache-2.0 & \textit{Custom} & the Pile, Wudao, \newline \textit{Crowdsourced} & PubMed,  Wikipedia, \newline arXiv, GitHub, etc. \\ \hline

        Llama2 & Text Generation &  \textit{Unknown} & Llama2 Community License & Llama2 Community License & \textit{Unknown} & n/a \\ \hline

        StarCoder & Text Generation &  \textit{Mixture} & Apache-2.0 & BigCode-OpenRAIL-M & The Stack & none \\ \hline

        Falcon-40B & Text Generation & ODC-By & Apache-2.0 & Apache-2.0 & RefinedWeb & Wikipedia, Reddit, \newline StackOverflow, etc. \\ \hline

        Waifu Diffusion & Text to Image & \textit{Mixture} & \textit{Unknown} & CreativeML-OpenRAIL-M & \textit{Unknown} & n/a \\ \hline

        Dolly-v2-12B & Text Generation & CC-BY-SA-3.0\&4.0 & MIT & MIT & databricks-dolly\newline-15k, the Pile & PubMed,  Wikipedia, \newline arXiv, GitHub, etc. \\ \hline

        Dreamlike Photoreal & Text to Image & \textit{Unknown} & \textit{Unknown} & \textit{Modified} CreativeML-\newline OpenRAIL-M & \textit{Unknown} & n/a \\ \hline

        Counterfeit & Text to Image & \textit{Unknow} & \textit{Unknown} & CreativeML-OpenRAIL-M & \textit{Unknown} & n/a \\ \hline

        GPT-2 & Text Generation & \textit{Mixture} & \textit{Modified} MIT & \textit{Modified} MIT & \textit{Crowdsourced} & WordPress, GitHub, \newline wikiHow, IMDb, etc. \\ \hline

        GPT-J-6B & Text Generation & \textit{Mixture} & Apache-2.0 & Apache-2.0 & the Pile & PubMed,  Wikipedia, \newline arXiv, GitHub, etc. \\ \hline

        LLaMA-7B & Text Generation & \textit{Mixture} & \textit{Custom} & \textit{Custom} & \textit{Crowdsourced} & GitHub, arXiv, etc. \\ \hline

        BERT & Fill Mask & \textit{Mixture} & Apache-2.0 & Apache-2.0 & Book Corpus, \newline Wikipedia (en) & Wikipedia (en) \\ \hline

        Whisper & ASR & \textit{Unknown} & MIT & MIT & \textit{Unknown} & n/a \\ \hline

        MPT & Text Generation & \textit{Mixture} & Apache-2.0 & Apache-2.0 & \textit{Crowdsourced} & Common Crawl, \newline Wikipedia, etc. \\ \hline
    
    % <TOTAL>
    % Data License: All Permissive(2); Risk of Copyleft/Proprietary(11); Custom/Modified(0); Unknown(6);
    % Software License: Permissive(11); Risk of Copyleft/Proprietary(n/a); Custom/Modified(2); Unknown(6);
    % Model License: Permissive(15); Risk of Copyleft/Proprietary(n/a); Custom/Modified(4); Unknown(0);

    \end{tabular}
\end{table*}

\section{Background and Related Work}
\label{sec:related}
In this section, we present the motivations for this work by introducing the background and prior related studies.

\subsection{Machine Learning Project Licensing}
% 现在 ML 的license的现状,包括数据的license问题
Typically, a ML project is constructed with data, software and models, which are usually governed by different licensing frameworks.
To profile current ML licensing, we summary licensing details for ML projects with over 1,000 likes available in Huggingface\footnote{https://huggingface.co/. Projects in same series but different versions are omitted.} model repository (See Appendix~\ref{apdx:A}).
Due to a lack of license management in development, we have to manually collect the license information from Huggingface, GitHub, related websites and publications.

\textbf{Data Licensing in ML}.
Based on our profile, half of ML projects claim their data is licensed in a mixture manner.
Additionally, 25\% of projects use a single dataset with a standard data license like Creative Commons (CC) licenses.
The data source of remaining projects (25\%) is unknown.
Obviously, legal compliance cannot be guaranteed when using data from unknown sources. 
However, there is also potential risk associated with using data under a mixture of licenses or a single license based on follow reasons:

First, the mixture of data sources may involve content under copyleft, non-public, and non-commercial licenses. 
We investigated the sources of mixture and found that only one dataset, the Pile~\cite{gao2020the}, explicitly removed non-permissive content.
Common sources of risk include Wikipedia, arXiv, PubMed and Common Crawl~\cite{henderson2023foundation} (See Table.~\ref{tab:works} for more examples).
For instance, sharing derivatives based on non-public licensed content raises suspicion of a license violation, and integrating copyleft content also poses a risk of license incompatibility conflicts.
Furthermore, some content sources like IMDb explicitly prohibit data mining in their \textit{Conditions of Use}.\footnote{"You may not use data mining, robots, screen scraping, or similar data gathering and extraction tools on this site, ..."}

Secondly, the single data license assigned by data collectors may be invalid.
In our profile, all datasets with a single license contain risky data sources.
Rajbahadur \textit{et al.}~\cite{rajbahadur2021can} investigated the sources of six public datasets and shown their inherent incompatibility for commercial use.
A real case is the copyright infringement lawsuit filed by Getty Images Inc., alleging that Stability AI Ltd. misused Getty Images photos to train its Stable Diffusion~\cite{rombach2022high} generative model (1:23-cv-00135).
However, the claimed license of training dataset~\cite{schuhmann2022laion} used for Stable Diffusion is CC-BY-4.0, which is a permissive license allowing for commercial use.
This highlights that ML data licensing is currently irregular and has become a significant factor in legal non-compliance.
Although Benjamin et al.~\cite{benjamin2019towards} have proposed the Montreal Data License (MDL) to foster fair use of data in AI activities, unfortunately, none of the ML projects adopted this license as shown in our profile.

\textbf{Software Licensing in ML.}
Distinct from OSS projects, only 50\% of ML projects release their code with standard OSS licenses.
About one-third of ML projects do not declare the code license (but have a model license), which is much higher than in OSS projects~\cite{cui2023empirical}.
Other projects switch to using AI model or custom licenses to insert additional disclaimers and restrictions related to AI activities, thereby increasing the diversity of licenses in this context.
However, given that ML, especially Neural Networks (NNs), is still in its emerging stages, the license dependency chain is shorter compared to OSS projects~\cite{buchkova2022dasea}, and most of them use the latest versions of OSS licenses like Apache-2.0 and MIT.

% 加一个license的统计数据?

\textbf{Model Licensing in ML.}
In contrast to software licensing, all ML projects have declared their model licenses.
The most popular license is Open Responsible AI License (OpenRAIL)~\cite{contractor2022behavioral}, which is a permissive license but includes copyleft-style use-based restrictions governing the use of the model and its derivatives.
There are 35\% of projects that insist on using unmodified OSS licenses for model licensing, even though these licensing language incurs conceptual ambiguities in the ML context.
An interesting finding is that, despite their training data being suspected to contain non-public content, the models are declared as free and open work~\cite{henderson2023foundation}.

\vspace{-2mm}
\begin{tcolorbox}
\textbf{Summary}.
ML project licenseing exhibit the following characteristics:
1) Ambiguous, unaccredited and over-permissive license declarations;
2) Emerging RAIL options for model licensing;
3) Unique license dependency structures in ML-specific components reusing.
There is a need for new methods to assess ML license compliance.
\end{tcolorbox}


% 分歧:自定义license,使用 content license (LayoutLMv3, i2vgen)

\subsection{OSS License Assessment}
License analysis for OSS projects has been extensively researched, but it's relatively unexplored in ML context.
The research scope and problems of OSS and ML license analysis can be classified into three tiers as shown in Figure~\ref{fig:pyramid}.
%, which can be divide into code level and software level.
%The code level mainly focus on the compliance of a single file or lines of code.
For instance, German \textit{et al.}~\cite{german2010sentence} proposed a sentence-based matching tool to identify the license of code.
Building on this work, Wu \textit{et al.}~\cite{wu2017analysis} further studied inconsistent changes among code clones through provenance analysis.
In addition to license identification~\cite{jaeger2017fossology}, Vendome et al.~\cite{vendome2017machine} proposed a ML-based clustering method to detect license exceptions.
These studies mainly deal with copyright issues at the code lines level, located in bottom tier of Figure~\ref{fig:pyramid}, which can be mapped to similar ML problems: finding the provenance of data sources~\cite{rajbahadur2021can} or modules~\cite{chen2022copy}.
However, these OSS tools perform software composition analysis through pattern matching or file scanning~\cite{ombredanne2020free}, which are not suitable to datasets and models that typically lack clear provenance and textual licenses.

Shifting the focus to the middle tier, there are some studies that explore license compatibility and violations in software packages~\cite{mathur2012empirical, wu2015method}. 
Kapitsaki et al.~\cite{kapitsaki2017automating} used Software Package Data Exchange (SPDX) files to detect conflicts in license compatibility (e.g., GPL-2.0 to GPL-3.0).
Cui \textit{et al.}~\cite{cui2023empirical} directly extracted terms from license texts using Natural Language Processing (NLP) to analyze license conflicts in OSS projects.
\textbf{However}, OSS license analysis works exhibit clear limitations when extended to ML projects.
Firstly, they lack support for dataset and model licenses. For example, RAILs and CC licenses are not listed in the SPDX. 
Secondly, the mixed use of licenses in current ML projects makes it challenging to interpret license conditions across different frameworks.
Lastly, these works only consider combinations and links in their analysis, whereas ML reuse involves a nested and iterative workflow with a more complex dependency structure (e.g., fine-tuning, embedding).

Distinct with previous studies, the research scope of our work is located in top and middle tiers.
we propose a practical tool ModelGo to assess potential license violations and non-granting righs errors in ML context.
We hope that ModelGo can assist developers in comprehending their obligations and risks when reusing ML components with multiple licenses~\cite{almeida2017software}, providing insights for constructing compaliant ML systems.


\begin{figure}[t]
    \centering
    \includegraphics[width=\linewidth]{fig/pyramid.pdf}
    \caption{Research scope and problems of ModelGo compare with traditional OSS license analysis.}
    \Description{}
    \label{fig:pyramid}
    \vspace{-5mm}
\end{figure}

%\subsection{Machine Learning IP Protection}
% 检测是否transfer knowledge,与本工作不冲突


\begin{comment}
    There is no consensus on whether the use of copyright works as input to train an AI system is an exercise of an exclusive right.
    There remains significant legal uncertainty about whether copyright applies to AI training, which means it may not always be clear whether a CC license applies.
    The larger model was trained on 256 cloud TPU v3 cores. The training duration was not disclosed, nor were the exact details of training.
    
    Open source software license compliance~\cite{ombredanne2020free}
    
    The open source definition~\cite{perens1999open}
    
    AFL~\cite{rosen2005open}
    
    Wudao2.0 1.75T MoE
    [FASTMOE: A FAST MIXTURE-OF-EXPERT TRAINING SYSTEM]
    [GLM-130B: AN OPEN BILINGUAL PRE-TRAINED MODEL]
    
    Objectives and challenges associated with analyzing dataset license compliance?
    Getty Images (US), Inc. v. Stability AI, Inc. (1:23-cv-00135)
    Andersen et al v. Stability AI Ltd. et al (3:23-cv-00201)
    We are not aware of any copyright restrictions of the material
    
    C4, Pile Common Crawl
    crowdsourced
    
    COCO (CC-BY 4.0), CIFAR10 -> Flickr
    Unsplash License \textit{Custom}: Compiling photos from Unsplash to replicate a similar or competing service. https://unsplash.com/license
    Pixabay License: Data mining, extraction, scraping and the use of programs or robots for automatic data collection and/or extraction of digital data on the Services and/or the content available therein is strictly prohibited for all purposes, including without limitation for machine learning purposes.
    
    Google Street View (SVHN) https://about.google/brand-resource-center/products-and-services/geo-guidelines/
    
    %实际上,license analysis应该可分为四层,最底层是license 是否 compliance (或者叫做dataset provenance extraction, License identification),例如数据集的license是否可以覆盖每个sample,还有就是distribution的版本的license和official source的license不一致,第二层是license之间是否conflict,第三层是应用所需的rights是否具备,最顶层是应用场景是否符合法律法规(regulation)
    
    
    Software reuse is very simple from the legal point of view, if a company or an
    individual reuses software for which it has copyrights. However, things change dramatically
    if one wants to reuse software made by others, since software is protected
    by copyright and possibly by patents. Without explicit permission, no person other
    than the copyright holder is allowed to copy, distribute, or make derivative works
    from the original work.
\end{comment}

\begin{comment}
\begin{table*}[t]
    \caption{Summary of licensing details for machine learning projects with over 1K likes on Huggingface. }
    \footnotesize
    \label{tab:MLP}
    \begin{tabular}{|p{2.1cm}|p{1.6cm}|p{2cm}|p{2.75cm}|p{3cm}|p{1.7cm}|p{2cm}|}
        \hline
        \textbf{ML Project} & \textbf{Task} & \textbf{Data License} & \textbf{Software License} & \textbf{Model License} & \textbf{Dataset} & \textbf{Risk Resource} \\ \hline
        
        Stable Diffusion v1-5 & Text to Image & CC-BY-4.0 & CreativeML-OpenRAIL-M & CreativeML-OpenRAIL-M & LAION-5B & Common Crawl \\ \hline
        
        BLOOM & Text Generation & \textit{Mixture} & \textit{Unknown} & BigScience-BLOOM-RAIL-1.0 & \textit{Crowdsourced} & Common Crawl, \newline Wikipedia, etc. \\ \hline

        OrangeMixs & Text to Image & \textit{Mixture} & \textit{Unknown} & CreativeML-OpenRAIL-M & \textit{Crowdsourced} & Danbooru \\ \hline

        ControlNet & Text to Image &  \textit{Unknown} & Apache-2.0 & OpenRAIL & \textit{Unknown} & n/a \\ \hline

        Openjourney & Text to Image &  CC-BY-NC-4.0 & \textit{Unknown} & CreativeML-OpenRAIL-M & Midjourney Gen & Midjourney Gen \\ \hline

        ChatGLM-6B & Text Generation &  \textit{Mixture} & Apache-2.0 & \textit{Custom} & the Pile, Wudao, \newline \textit{Crowdsourced} & PubMed,  Wikipedia, \newline arXiv, GitHub, etc. \\ \hline

        Llama2 & Text Generation &  \textit{Unknown} & Llama2 Community License & Llama2 Community License & \textit{Unknown} & n/a \\ \hline

        StarCoder & Text Generation &  \textit{Mixture} & Apache-2.0 & BigCode-OpenRAIL-M & The Stack & none \\ \hline

        Falcon-40B & Text Generation & ODC-By & Apache-2.0 & Apache-2.0 & RefinedWeb & Wikipedia, Reddit, \newline StackOverflow, etc. \\ \hline

        Waifu Diffusion & Text to Image & \textit{Mixture} & \textit{Unknown} & CreativeML-OpenRAIL-M & \textit{Unknown} & n/a \\ \hline

        Dolly-v2-12B & Text Generation & CC-BY-SA-3.0\&4.0 & MIT & MIT & databricks-dolly\newline-15k, the Pile & PubMed,  Wikipedia, \newline arXiv, GitHub, etc. \\ \hline

        Dreamlike Photoreal & Text to Image & \textit{Unknown} & \textit{Unknown} & \textit{Modified} CreativeML-\newline OpenRAIL-M & \textit{Unknown} & n/a \\ \hline

        Counterfeit & Text to Image & \textit{Unknow} & \textit{Unknown} & CreativeML-OpenRAIL-M & \textit{Unknown} & n/a \\ \hline

        GPT-2 & Text Generation & \textit{Mixture} & \textit{Modified} MIT & \textit{Modified} MIT & \textit{Crowdsourced} & WordPress, GitHub, \newline wikiHow, IMDb, etc. \\ \hline

        GPT-J-6B & Text Generation & \textit{Mixture} & Apache-2.0 & Apache-2.0 & the Pile & PubMed,  Wikipedia, \newline arXiv, GitHub, etc. \\ \hline

        LLaMA-7B & Text Generation & \textit{Mixture} & \textit{Custom} & \textit{Custom} & \textit{Crowdsourced} & GitHub, arXiv, etc. \\ \hline

        BERT & Fill Mask & \textit{Mixture} & Apache-2.0 & Apache-2.0 & Book Corpus, \newline Wikipedia (en) & Wikipedia (en) \\ \hline

        Whisper & ASR & \textit{Unknown} & MIT & MIT & \textit{Unknown} & n/a \\ \hline

        MPT & Text Generation & \textit{Mixture} & Apache-2.0 & Apache-2.0 & \textit{Crowdsourced} & Common Crawl, \newline Wikipedia, etc. \\ \hline
    
    % <TOTAL>
    % Data License: All Permissive(2); Risk of Copyleft/Proprietary(11); Custom/Modified(0); Unknown(6);
    % Software License: Permissive(11); Risk of Copyleft/Proprietary(n/a); Custom/Modified(2); Unknown(6);
    % Model License: Permissive(15); Risk of Copyleft/Proprietary(n/a); Custom/Modified(4); Unknown(0);

    \end{tabular}
\end{table*}

\end{comment}

\section{Method}
This section is organized around three key questions in the context of ML license analysis: (i) How to determine the corresponding conditions in licenses for certain model reuse mechanisms? (ii) How to capture the dependency structure of a machine learning project? (iii) What types of non-compliance exist in ML projects and how to assess them?
We will present our solutions to these questions in the following sections.

\subsection{Taxonomy for ML License Analysis}
Determining the corresponding conditions in licenses is a challenging task for ML projects due to the conceptual ambiguities in existing licensing language and the disorganization in current ML licensing practices.
For example, CC-BY-ND prohibits the sharing of derivatives of licensed materials.
However, its definition of making derivatives is unclear in the context of ML domain.
For instance, should embeddings of a corpus be considered a derivative work upon that corpus?
Unfortunately, even though Creative Commons provides a flow chart to illustrate the trigger conditions of CC licenses in the context of AI activity~\cite{creative2023artificial}, it raises another question: \textit{Is the output considered protectable copyright subject matter?}
The answer depends on how the embedding activity is interpreted, for example, considering it as a translation of the original work can trigger the CC license.

MDL advocates the use of a "Top Sheet" to delineate what ML activities are allowed with data~\cite{benjamin2019towards}, but this proposal is rarely implemented in practice (things would be easier if it were widely accepted). 
Making things more complex, some projects release their models under free content licenses, like LayoutLMv3 model~\cite{huang2022layoutlmv3}, which is licensed under CC-BY-NC-SA-4.0. 
This disorganization makes it unclear what kinds of ML activities can trigger licenses conditions in different contexts.
An ideal and elegant solution would be to encourage licensors to make context-appropriate adaptations in their license agreements or terms of use to clarify the granted rights related to ML activities. 
However, some ML components may be composed of prior works that are shared under copyleft license templates, which may disallow such relicensing of their derivatives to a new license.
Therefore, it is necessary to establish practical rules to bridge AI activities and existing licensing language.

To address the above challenge, we propose a result-based taxonomy that categorizes all AI activities into four categories based on the forms of their results. 
In our taxonomy, there are four categories of AI activities: Combination, Amalgamation, Distillation, and Generation, which are defined by four forms of their results, respectively: (1) Combination with strong separation; (2) Combination with weak separation; (3) Derivatives from concepts; and (4) Derivatives from data.
Correspondingly, we can also categorize the usage behaviors in license language into these four categories based on their outcome forms.
We leverage Figure~\ref{fig:tax} to illustrate this idea, and the details of the four categories are as follows:

\textbf{Combination}


\begin{figure}[t]
    \centering
    \includegraphics[width=\linewidth]{fig/taxonomy.pdf}
    \caption{Our proposed taxonomy bridging AI activities and license terms based on their result forms.}
    \Description{}
    \label{fig:tax}
\end{figure}

% 仅对于模糊定义而言,如果有明确定义那么优先级更高。 存在对应多种解释的情况,存在不同license的解释不通的情况  Train 的解释, 取决于对象
translated, altered, arranged, transformed, or otherwise modified 


Licensing Language Requires Standardization and  to ML and AI
the notion of derivative work is ill defined
conceptual ambiguities in existing licensing language
There is no consensus on whether the use

\begin{comment}
If CC SA-licensed content is included in a database, does the entire database have to be licensed under an SA license?

If CC SA-licensed content is included in a database, does the entire database have to be licensed under an SA license?
CC licenses never require a reuser of a CC-licensed work to make the original work or resulting works (collections, derivatives, etc.) publicly available. There are lots of private reuses of works that are permitted by CC’s licenses that do not require compliance with their terms. Regarding ShareAlike, the condition only applies if a work is modified and if the work is shared publicly. In the situation where a reuser created a dataset of photos and made it publicly available, and assuming copyright permission is required, then what is released is likely a collection or compilation of pre-existing works. CC licenses do not require the collection or the compilation itself to be made available under an SA license, even though each individual work is still licensed individually under an SA license and if they were modified by the distributor the modified photo would need to be licensed under the same terms. For example, were Creative Commons to compile photographs from a photo sharing website under a BY-SA 2.0 license and create a database that it then publicly distributed, CC could license the collection as a whole under a BY license, but the photographs would continue to be licensed under BY-SA 2.0.
\end{comment}

\section{Case Study Details}
An ideal practice of Modelgo is to assess real-world ML projects and detect their potential license compliance issues. 
However, this can be challenging in practice due to three present situations:

(1) Prevalent Licensing Disorganization in ML Projects: Many ML projects lack organized licensing information, making it difficult to ascertain the licenses of individual components.

(2) Lack of Development Lifecycle Information for ML Reusing: ML reusing often occurs without a clear record, making it hard to trace the origins and licenses of components used.

(3) Non-compliance within Datasets: Crowdsourced datasets often suffer from license non-compliance issues~\cite{rajbahadur2021can}, making the licenses (usually permissive) declared by dataset collectors invalid.

Consequently, directly analyzing real-world ML projects may result in uncertainty, over-optimistic results, and often fail to detect any license conflicts.
Therefore, to validate Modelgo, we have designed five ML scenarios rendered using 15 common data sources and 11 models that cover 5 modalities and 7 tasks, respectively.
Table~\ref{tab:works} shows the specifications of the involved data sources and models, whose licenses include copyleft, permissive, public domain, and no public license~\footnote{Some data sources contain crowdsourced content with multiple licenses, and we selected a non-public domain license among them.}.
Furthermore, our case studies can cover all events listed in Table~\ref{tab:analysis}, and the their details and findings are provided in the following section.

\begin{table}[t]
    \caption{Specifications of AI components used in case studies, which include \textcolor{Copyleft}{Copyleft License}, \textcolor{Permissive}{Permissive License}, \textcolor{Public}{Public Domain Licens} and No Public License.}
    \footnotesize
    \label{tab:works}
    \begin{tabular}{|p{1.6cm}|p{3cm}|p{0.6cm}|p{1.7cm}|}
        \hline
        \rowcolor[gray]{.8}
        \textbf{Work Name} & \textbf{License Name} & \textbf{Type} & \textbf{Modality/Usage}  \\ \hline
        Wikipedia & \textcolor{Copyleft}{CC-BY-SA-4.0} & \multirow{15}{*}{Data} & \multirow{6}{*}{Text}   \\ \cline{1-2}
        StackExchange & \textcolor{Copyleft}{CC-BY-SA-4.0}  &  &    \\ \cline{1-2}
        FreeLaw & CC-BY-ND-4.0 &  &   \\ \cline{1-2}
        arXiv & \textcolor{Copyleft}{CC-BY-NC-SA-4.0} &  &   \\ \cline{1-2}
        PubMed & \textcolor{Copyleft}{CC-BY-NC-SA-4.0} &  &    \\ \cline{1-2}
        Deep-sequoia & CC-BY-NC-ND-4.0 &  &   \\ \cline{1-2} \cline{4-4}

        Midjourney Gen & CC-BY-NC-ND-4.0 &  & \multirow{6}{*}{Image}  \\ \cline{1-2}
        Flickr & \textcolor{Copyleft}{CC-BY-NC-SA-4.0} &  &   \\ \cline{1-2}
        StockSnap & \textcolor{Public}{CC0-1.0} &  &   \\ \cline{1-2}
        Wikimedia & \textcolor{Copyleft}{CC-BY-SA-4.0} &  &   \\ \cline{1-2}
        OpenClipart & \textcolor{Public}{CC0-1.0} &  &   \\ \cline{1-2} \cline{4-4}
        
        ccMixter & \textcolor{Permissive}{CC-BY-NC-4.0} & & \multirow{2}{*}{Voice}  \\ \cline{1-2}
        Jamendo & CC-BY-NC-ND-4.0 &  &   \\ \cline{1-2} \cline{4-4}
        
        Thingverse & \textcolor{Copyleft}{CC-BY-NC-SA-4.0} &  & 3D model  \\ \cline{1-2} \cline{4-4}

        Vimeo & CC-BY-NC-ND-4.0 &  & Video  \\ \hline

        Baize & \textcolor{Copyleft}{GPL-3.0} & \multirow{11}{*}{Model} & \multirow{4}{*}{Text Generation}   \\ \cline{1-2}
        BLOOM & \textcolor{Permissive}{BigScience-BLOOM-RAIL-1.0} & &   \\ \cline{1-2}
        Llama2 & \textcolor{Permissive}{Llama2} & &   \\ \cline{1-2}
        BigTranslate & \textcolor{Copyleft}{GPL-3.0} & &   \\ \cline{1-2} \cline{4-4}

        BERT & \textcolor{Permissive}{Apache-2.0} &  & Fill-Mask   \\ \cline{1-2} \cline{4-4}

        Stable Diffusion & \textcolor{Permissive}{CreativeML-OpenRAIL-M} & & Text to Image  \\ \cline{1-2} \cline{4-4}

        MaskFormer & \textcolor{Permissive}{CC-BY-NC-4.0} & & Image  \\ \cline{1-2}
        DETR & \textcolor{Permissive}{Apache-2.0} & & Segmentation \\ \cline{1-2} \cline{4-4}

        Whisper & \textcolor{Permissive}{MIT} & & Voice to Text  \\ \cline{1-2} \cline{4-4}

        X-Clip & \textcolor{Permissive}{MIT} & & Video to Text  \\ \cline{1-2} \cline{4-4}

        I2VGen-XL & CC-BY-NC-ND-4.0 & & Image to Video  \\ \hline

        \end{tabular}
\end{table}

\subsection{CASE \Romannum{1} : Corpus Combination}

Our first case is corpus combination, which is very common in crowdsourced LLM datasets~\cite{gao2020the, penedo2023refinedweb, kocetkov2023stack}. 
Additionally, we also consider scenarios where the corpus is extended with the help of translation LLM.
As shown in Figure~\ref{fig:case1} (a), we first translate\footnote{In our cases, we treat translation as a specific form of embedding with a natural language output.} \textit{arXiv} and \textit{Stack Exchange} using \textit{Big Translate} model, then we combine these translated corpuses with \textit{Deep-sequoia} and \textit{FreeLaw}.
This combined corpus is the final work, intended for commercial purposes.
Figure~\ref{fig:case1} (b) depicts a variation in which the final work is a combination of translated corpus and the LLM.
Note that, to simplify analysis, we treat these no public licenses, such as CC-BY-ND-4.0 and CC-BY-NC-ND-4.0, as permissive licenses with limitations on sharing derivatives, as they do not include any copyleft terms.
%We have also consolidated some redundant results reported by intermediate reused components to simplify our figures.
The interpretation of license analysis results is as follows:

\boxed{\text{Results of CASE \Romannum{1} (a)}} The copyleft conditions about \textit{translation} of the CC licenses were triggered, which means that the translated corpuses are also covered by the original licenses.
As a result, the translated \textit{arXiv} and \textit{Stack Exchange} corpuses remain under the original copyleft license. 
However, combining these corpuses with another copyleft-licensed \textit{Deep-sequoia} corpus did not result in the multiple copyleft licenses error, as the combination with strong separate falls outside the proliferate coverage of LGPL-LR.
The proliferation extended to the final work and force it to be licensed under LGPL-LR as well.
It is important to note that only the effort taken to combine the corpuses is under LGPL-LR, and the licensing action to the final work will not change the licenses of its components.

In this case, there are two types of errors according to Modelgo's assessment.
The first error arises from the CC-BY-NC-SA-4.0 license of the translated \textit{arXiv}, which doesn't grant the right of commercial use\footnote{This error also arises from \textit{arXiv} since it is a sub-work of the translated \textit{arXiv}, we will consolidate this type of redundant in the rest of the case studies.}. 
The second error is caused by the fact that the redistribution rights of FreeLaw are not granted to comply with CC-BY-ND-4.0.
There are also many restrictions, such as the final work must state the changes compared to the original work and must retain the licenses and notice files of the original works.
In addition, Modelgo also indicates that the granted rights of LGPL-LR are revocable, which poses a potential risk for further redistribution.

\boxed{\text{Results of CASE \Romannum{1} (b)}} Different from CASE \Romannum{1} (a), the final work in CASE \Romannum{1} (b) is licensed under another copyleft license GPL-3.0 from \textit{Big Translate}.
This is because LGPL-LR has a license proliferation exemption for reused results that are no longer classified as linguistic resources.
Consequently, the license of final work is proliferated by GPL-3.0.
Additionally, besides the rights not granted error arising from CC-BY-NC-ND-4.0, this license also explicitly prohibits any form of sharing derivatives, resulting in a cannot share error.

\begin{figure}[t]
    \centering
    \includegraphics[width=\linewidth]{fig/case1.pdf}
    \caption{CASE \Romannum{1}: Corpus Combination. (a) Example of copyleft proliferation rules ; (b) LGPL-LR no linguistic resource exemption, CC No redistribution. AI ctivities: \boxed{\text{E}}mbed, \boxed{\text{C}}ombine.}
    \Description{}
    \label{fig:case1}
\end{figure}

\begin{figure}[t]
    \centering
    \includegraphics[width=\linewidth]{fig/case2.pdf}
    \caption{Case Study \Romannum{2}: Mixture of Experts. (a) BLOOM-RAIL, binary of GPL; (b) Unlicense, CC-BY-NC no distribute derivative. GPL Automatic Licensing of Downstream Recipients}
    \Description{}
    \label{fig:case2}
\end{figure}

\begin{figure}[t]
    \centering
    \includegraphics[width=\linewidth]{fig/case3.pdf}
    \caption{Case Study \Romannum{3}: Pipeline.}
    \Description{}
    \label{fig:case3}
\end{figure}

% Responsible AI 的 copyleft use behavioral 问题:历史累积

\begin{figure}[t]
    \centering
    \includegraphics[width=\linewidth]{fig/case4.pdf}
    \caption{Case Study \Romannum{4}: distillation and model averaging.}
    \Description{}
    \label{fig:case4}
\end{figure}

\begin{figure}[t]
    \centering
    \includegraphics[width=\linewidth]{fig/case5.pdf}
    \caption{Case Study \Romannum{5}: distillation and model averaging.}
    \Description{}
    \label{fig:case5}
\end{figure}

\section{Conclusion}
\label{sec:conclusion}
Component reusing is prevalent in today's ML project development lifecycle, yet legal compliance issues are often ignored. Furthermore, it can be challenging for developers to understand elusive license terms and identify the potential risk of license violations.
Therefore, given the particularity of ML projects and licensing practices, we propose a practical license analysis tool to analyze their license conflicts.
We leverage five case studies to demonstrate the feasibility of our method, and our findings provide constructive guidelines to minimize conflicts.


%\section{Disclaimer}
%The content presented in this article is intended for general informational purposes only and should not be construed as legal advice. Any views, opinions, findings, conclusions, or recommendations expressed in this material are the sole responsibility of the author(s) and do not represent the perspectives of any organization or entity.


%%
%% The acknowledgments section is defined using the "acks" environment
%% (and NOT an unnumbered section). This ensures the proper
%% identification of the section in the article metadata, and the
%% consistent spelling of the heading.

\begin{acks}
This research is supported by the National Research Foundation Singapore and DSO National Laboratories under the AI Singapore Programme (AISG Award No: AISG2-RP-2020-018). 
Any opinions, findings and conclusions or recommendations expressed in this material are those of the authors and do not reflect the views of National Research Foundation, Singapore.
\end{acks}



%%
%% The next two lines define the bibliography style to be used, and
%% the bibliography file.
\bibliographystyle{ACM-Reference-Format}
%\balance
\bibliography{REF}

%%
%% If your work has an appendix, this is the place to put it.
\newpage
\appendix
\section{Appendix}

\subsection{DISCLAIMER}
\label{apdx:disclaimer}
The content presented in this article is intended for general informational purposes only and should not be construed as legal advice. Any views, opinions, findings, conclusions, or recommendations expressed in this material are the sole responsibility of the author(s) and do not represent the perspectives of any organization or entity.

\subsection{Additional Figure and Table}
\label{apdx:A}

Figure~\ref{fig:flowchart} illustrates the flowchart for minimizing license conflicts in a ML project.
Table~\ref{tab:list} lists the licenses supported by ModelGo.
Table~\ref{tab:works} shows the specifications of the involved data sources and models in the case studies.
Table~\ref{tab:stats} presents statistical data related to licenses and their corresponding count of works on Huggingface.
Table~\ref{tab:MLP} displays the summary of licensing details for ML projects with over 1K likes on Huggingface (\url{https://huggingface.co/}, projects in same series but different versions are omitted).

\begin{figure}[H]
  \centering
  \includegraphics[scale=1]{fig/flowchart.pdf}
  \caption{Flowchart for minimizing license conflicts in ML projects.}
  \Description{}
  \label{fig:flowchart}
\end{figure}

\clearpage

\begin{table}[h]
  \caption{List of licenses (represented by SPDX short IDs) supported by ModelGo, covering over 96\% of licensed models and datasets on Huggingface.}
  %\vspace{-3mm}
  \scriptsize
  \label{tab:list}
  \begin{tabular}{|p{2.45cm}|p{2.45cm}|p{2.45cm}|}
  \hline
  \rowcolor[gray]{.8}
  \textbf{OSS License} (99.8\%) & \textbf{Content License} (96.6\%) & \textbf{AI Model License} (98.2\%)\\ \hline
  Apache-2.0, Unlicense, MIT, AFL-3.0, GPL-3.0, AGPL-3.0, LGPL-3.0, LGPL-2.1, BSD-3-Clause, BSD-3-Clause-Clear, BSD-2-Clause, Artistic-2.0, WTFPL-2.0, OSL-3.0, ECL-2.0
  &
  CC0-1.0, CC-BY-4.0, CC-BY-SA-4.0, CC-BY-NC-4.0, CC-BY-ND-4.0, CC-BY-NC-ND-4.0, CC-BY-NC-SA-4.0, PDDL, C-UDA, LGPL-LR, GFDL
  & 
  OpenRAIL++, CreativeML-OpenRAIL-M, BigScience-BLOOM-RAIL-1.0, Llama2, OPT-175B, SEER
  \\ \hline

  \end{tabular}
  %\vspace{-2mm}
\end{table}

\begin{table}[h]
  \caption{Specifications of AI components used in case studies, which include \textcolor{Copyleft}{Copyleft License}, \textcolor{Permissive}{Permissive License}, \textcolor{Public}{Public Domain License} and Non-Public License.}
  \vspace{-1mm}
  \footnotesize
  \label{tab:works}
  \begin{tabular}{|p{1.6cm}|p{3cm}|p{0.6cm}|p{1.7cm}|}
      \hline
      \rowcolor[gray]{.8}
      \textbf{Work Name} & \textbf{License Name} & \textbf{Type} & \textbf{Modality/Usage}  \\ \hline
      Wikipedia & \textcolor{Copyleft}{CC-BY-SA-4.0} & \multirow{15}{*}{Data} & \multirow{6}{*}{Text}   \\ \cline{1-2}
      StackExchange & \textcolor{Copyleft}{CC-BY-SA-4.0}  &  &    \\ \cline{1-2}
      FreeLaw & CC-BY-ND-4.0 &  &   \\ \cline{1-2}
      arXiv & \textcolor{Copyleft}{CC-BY-NC-SA-4.0} &  &   \\ \cline{1-2}
      PubMed & \textcolor{Copyleft}{CC-BY-NC-SA-4.0} &  &    \\ \cline{1-2}
      Deep-sequoia & CC-BY-NC-ND-4.0 &  &   \\ \cline{1-2} \cline{4-4}

      Midjourney Gen & CC-BY-NC-ND-4.0 &  & \multirow{6}{*}{Image}  \\ \cline{1-2}
      Flickr & \textcolor{Copyleft}{CC-BY-NC-SA-4.0} &  &   \\ \cline{1-2}
      StockSnap & \textcolor{Public}{CC0-1.0} &  &   \\ \cline{1-2}
      Wikimedia & \textcolor{Copyleft}{CC-BY-SA-4.0} &  &   \\ \cline{1-2}
      OpenClipart & \textcolor{Public}{CC0-1.0} &  &   \\ \cline{1-2} \cline{4-4}
      
      ccMixter & \textcolor{Permissive}{CC-BY-NC-4.0} & & \multirow{2}{*}{Voice}  \\ \cline{1-2}
      Jamendo & CC-BY-NC-ND-4.0 &  &   \\ \cline{1-2} \cline{4-4}
      
      Thingverse & \textcolor{Copyleft}{CC-BY-NC-SA-4.0} &  & 3D model  \\ \cline{1-2} \cline{4-4}

      Vimeo & CC-BY-NC-ND-4.0 &  & Video  \\ \hline

      Baize & \textcolor{Copyleft}{GPL-3.0} & \multirow{11}{*}{Model} & \multirow{4}{*}{Text Generation}   \\ \cline{1-2}
      BLOOM & \textcolor{Permissive}{BigScience-BLOOM-RAIL-1.0} & &   \\ \cline{1-2}
      Llama2 & \textcolor{Permissive}{Llama2 Community License} & &   \\ \cline{1-2}
      BigTranslate & \textcolor{Copyleft}{GPL-3.0} & &   \\ \cline{1-2} \cline{4-4}

      BERT & \textcolor{Permissive}{Apache-2.0} &  & Fill-Mask   \\ \cline{1-2} \cline{4-4}

      Stable Diffusion & \textcolor{Permissive}{CreativeML-OpenRAIL-M} & & Text to Image  \\ \cline{1-2} \cline{4-4}

      MaskFormer & \textcolor{Permissive}{CC-BY-NC-4.0} & & Image  \\ \cline{1-2}
      DETR & \textcolor{Permissive}{Apache-2.0} & & Segmentation \\ \cline{1-2} \cline{4-4}

      Whisper & \textcolor{Permissive}{MIT} & & Voice to Text  \\ \cline{1-2} \cline{4-4}

      X-Clip & \textcolor{Permissive}{MIT} & & Video to Text  \\ \cline{1-2} \cline{4-4}

      I2VGen-XL & CC-BY-NC-ND-4.0 & & Image to Video  \\ \hline

      \end{tabular}
\end{table}

\begin{table}[h]
  \caption{List of Huggingface supported licenses and work count, with ModelGo supported licenses highlighted in BOLD. Note that many works do not explicitly indicate their license version. (Accessed on October 11, 2023). }
  \scriptsize
  \label{tab:stats}
  \begin{tabular}{|ll||ll|}
  \hline
  \rowcolor[gray]{.8} 
  \multicolumn{2}{|c||}{Model (Total Work: 355,150)}     & \multicolumn{2}{c|}{Dataset (Total Work: 69,277)}   \\ \hline
  \rowcolor[gray]{.9} 
  \multicolumn{1}{|l|}{License Name} & Count & \multicolumn{1}{l|}{License Name} & Count \\ \hline
  \multicolumn{1}{|l|}{\textbf{Apache-2.0}} & 46,758 & \multicolumn{1}{l|}{MIT} & 5,415 \\ \hline
  \multicolumn{1}{|l|}{\textbf{MIT}} & 21,365 & \multicolumn{1}{l|}{Apache-2.0} & 3,026 \\ \hline %3020+6
  \multicolumn{1}{|l|}{OpenRAIL} & 17,760 & \multicolumn{1}{l|}{OpenRAIL} & 1,639 \\ \hline
  \multicolumn{1}{|l|}{\textbf{CreativeML-OpenRAIL-M}} & 12,059 & \multicolumn{1}{l|}{\textbf{CC-BY-4.0}} & 1,355 \\ \hline
  \multicolumn{1}{|l|}{other} & 6,521 & \multicolumn{1}{l|}{other} & 1,257 \\ \hline
  \multicolumn{1}{|l|}{CC-BY-NC-4.0} & 2,867 & \multicolumn{1}{l|}{\textbf{CC-BY-SA-4.0}} & 609 \\ \hline
  \multicolumn{1}{|l|}{CC-BY-4.0} & 2,676 & \multicolumn{1}{l|}{AFL-3.0} & 515 \\ \hline
  \multicolumn{1}{|l|}{\textbf{AFL-3.0}} & 2,111 & \multicolumn{1}{l|}{CC} & 444 \\ \hline
  \multicolumn{1}{|l|}{\textbf{Llama2}} & 1,776 & \multicolumn{1}{l|}{\textbf{CC0-1.0}} & 435 \\ \hline
  \multicolumn{1}{|l|}{CC-BY-NC-SA-4.0} & 1,312 & \multicolumn{1}{l|}{\textbf{CC-BY-NC-4.0}} & 385 \\ \hline
  \multicolumn{1}{|l|}{\textbf{GPL-3.0}} & 1,080 & \multicolumn{1}{l|}{\textbf{CC-BY-NC-SA-4.0}} & 378 \\ \hline
  \multicolumn{1}{|l|}{CC-BY-SA-4.0} & 959 & \multicolumn{1}{l|}{CC-BY-SA-3.0} & 377 \\ \hline
  \multicolumn{1}{|l|}{\textbf{OpenRAIL++}} & 667 & \multicolumn{1}{l|}{CreativeML-OpenRAIL-M} & 290 \\ \hline
  \multicolumn{1}{|l|}{CC} & 625 & \multicolumn{1}{l|}{GPL-3.0} & 266 \\  \hline
  \multicolumn{1}{|l|}{\textbf{BigScience-OpenAI-M}} & 596 & \multicolumn{1}{l|}{\textbf{CC-BY-NC-ND-4.0}} & 190 \\ \hline
  \multicolumn{1}{|l|}{\textbf{Artistic-2.0}} & 579 & \multicolumn{1}{l|}{BigScience-OpenRAIL-M} & 114 \\ \hline
  \multicolumn{1}{|l|}{\textbf{BSD-3-Clause}} & 525 & \multicolumn{1}{l|}{CC-BY-3.0} & 94 \\ \hline
  \multicolumn{1}{|l|}{\textbf{BigScience-BLOOM-RAIL-1.0}} & 422 & \multicolumn{1}{l|}{CC-BY-2.0} & 91 \\ \hline
  \multicolumn{1}{|l|}{\textbf{WTFPL}} & 331 & \multicolumn{1}{l|}{Artistic-2.0} & 91 \\ \hline
  \multicolumn{1}{|l|}{CC-BY-SA-3.0} & 288 & \multicolumn{1}{l|}{ODC-by} & 80 \\ \hline
  \multicolumn{1}{|l|}{CC0-1.0} & 270 & \multicolumn{1}{l|}{WTFPL} & 80 \\ \hline
  \multicolumn{1}{|l|}{\textbf{BigCode-OpenRAIL-M}} & 251 & \multicolumn{1}{l|}{Unlicense} & 68 \\ \hline
  \multicolumn{1}{|l|}{\textbf{AGPL-3.0}} & 237 & \multicolumn{1}{l|}{Llama2} & 63 \\ \hline
  \multicolumn{1}{|l|}{\textbf{Unlicense}} & 199 & \multicolumn{1}{l|}{BSD} & 62 \\ \hline
  \multicolumn{1}{|l|}{CC-BY-NC-ND-4.0} & 194 & \multicolumn{1}{l|}{GPL} & 54 \\ \hline
  \multicolumn{1}{|l|}{GPL} & 173 & \multicolumn{1}{l|}{\textbf{C-UDA}} & 49 \\ \hline
  \multicolumn{1}{|l|}{BSD} & 155 & \multicolumn{1}{l|}{AGPL-3.0} & 46 \\ \hline
  \multicolumn{1}{|l|}{CC-BY-3.0} & 104 & \multicolumn{1}{l|}{CC-BY-NC-SA-3.0} & 38 \\ \hline
  \multicolumn{1}{|l|}{GPL-2.0} & 84 & \multicolumn{1}{l|}{ODBL} & 35 \\ \hline
  \multicolumn{1}{|l|}{CC-BY-2.0} & 80 & \multicolumn{1}{l|}{\textbf{GFDL}} & 34 \\ \hline
  \multicolumn{1}{|l|}{BSL-1.0} & 75 & \multicolumn{1}{l|}{BSD-3-Clause} & 34 \\ \hline
  \multicolumn{1}{|l|}{\textbf{BSD-2-Clause}} & 74 & \multicolumn{1}{l|}{\textbf{CC-BY-ND-4.0}} & 32 \\ \hline
  \multicolumn{1}{|l|}{\textbf{LGPL-3.0}} & 65 & \multicolumn{1}{l|}{CC-BY-NC-3.0} & 28 \\ \hline
  \multicolumn{1}{|l|}{C-UDA} & 57 & \multicolumn{1}{l|}{BigScience-BLOOM-RAIL-1.0} & 28 \\ \hline
  \multicolumn{1}{|l|}{CC-BY-NC-2.0} & 48 & \multicolumn{1}{l|}{GPL-2.0} & 26 \\ \hline
  \multicolumn{1}{|l|}{CC-BY-NC-3.0} & 45 & \multicolumn{1}{l|}{OpenRAIL++} & 24 \\ \hline
  \multicolumn{1}{|l|}{\textbf{OSL-3.0}} & 44 & \multicolumn{1}{l|}{CC-BY-NC-2.0} & 21 \\ \hline
  \multicolumn{1}{|l|}{\textbf{ECL-2.0}} & 35 & \multicolumn{1}{l|}{BigCode-OpenRAIL-M} & 20 \\ \hline
  \multicolumn{1}{|l|}{PDDL} & 35 & \multicolumn{1}{l|}{\textbf{PDDL}} & 20 \\ \hline
  \multicolumn{1}{|l|}{\textbf{BSD-3-Clause-Clear}} & 28 & \multicolumn{1}{l|}{BSD-2-Clause} & 16 \\ \hline
  \multicolumn{1}{|l|}{CC-BY-ND-4.0} & 27 & \multicolumn{1}{l|}{LGPL-3.0} & 15 \\ \hline
  \multicolumn{1}{|l|}{GFDL} & 26 & \multicolumn{1}{l|}{CDLA-Sharing-1.0} & 14 \\ \hline
  \multicolumn{1}{|l|}{Ms-PL} & 26 & \multicolumn{1}{l|}{CC-BY-2.5} & 12 \\ \hline
  \multicolumn{1}{|l|}{Zlib} & 25 & \multicolumn{1}{l|}{Ms-PL} & 11 \\ \hline
  \multicolumn{1}{|l|}{LGPL} & 21 & \multicolumn{1}{l|}{CDLA-Permissive-2.0} & 11 \\ \hline
  \multicolumn{1}{|l|}{DeepFloyd-IF-License} & 19 & \multicolumn{1}{l|}{CC-BY-NC-SA-2.0} & 10 \\ \hline
  \multicolumn{1}{|l|}{CC-BY-NC-SA-3.0} & 19 & \multicolumn{1}{l|}{MPL-2.0} & 10 \\ \hline
  \multicolumn{1}{|l|}{LGPL-LR} & 17 & \multicolumn{1}{l|}{EUPL-1.1} & 10 \\ \hline
  \multicolumn{1}{|l|}{MPL-2.0} & 16 & \multicolumn{1}{l|}{CC-BY-NC-ND-3.0} & 10 \\ \hline
  \multicolumn{1}{|l|}{ISC} & 15 & \multicolumn{1}{l|}{BSL-1.0} & 10 \\ \hline
  \multicolumn{1}{|l|}{CC-BY-NC-SA-2.0} & 15 & \multicolumn{1}{l|}{BSD-3-Clause-Clear} & 8 \\ \hline
  \multicolumn{1}{|l|}{ODBL} & 15 & \multicolumn{1}{l|}{LGPL} & 6 \\ \hline
  \multicolumn{1}{|l|}{CC-BY-2.0} & 14 & \multicolumn{1}{l|}{ECL-2.0} & 6 \\ \hline
  \multicolumn{1}{|l|}{CC-BY-NC-ND-3.0} & 14 & \multicolumn{1}{l|}{OSL-3.0} & 5 \\ \hline
  \multicolumn{1}{|l|}{ODB-by} & 13 & \multicolumn{1}{l|}{ISC} & 5 \\ \hline
  \multicolumn{1}{|l|}{NCSA} & 9 & \multicolumn{1}{l|}{\textbf{LGPL-LR}} & 4 \\ \hline
  \multicolumn{1}{|l|}{EPL-2.0} & 9 & \multicolumn{1}{l|}{PostgreSQL} & 3 \\ \hline
  \multicolumn{1}{|l|}{EUPL-1.1} & 9 & \multicolumn{1}{l|}{Zlib} & 3 \\ \hline
  \multicolumn{1}{|l|}{CDLA-Sharing-1.0} & 7 & \multicolumn{1}{l|}{EPL-2.0} & 2 \\ \hline
  \multicolumn{1}{|l|}{\textbf{LGPL-2.1}} & 6 & \multicolumn{1}{l|}{OFL-1.1} & 2 \\ \hline
  \multicolumn{1}{|l|}{PostgreSQL} & 5 & \multicolumn{1}{l|}{LGPL-2.1} & 1 \\ \hline
  \multicolumn{1}{|l|}{LPPL-1.3c} & 5 & \multicolumn{1}{l|}{CDLA-Permissive-1.0} & 1 \\ \hline
  \multicolumn{1}{|l|}{EPL-1.0} & 4 & \multicolumn{1}{l|}{CC-BY-2.0} & 1 \\ \hline
  \multicolumn{1}{|l|}{OFL-1.1} & 3 & \multicolumn{1}{l|}{NCSA} & 1 \\ \hline
  \multicolumn{1}{|l|}{TII-Falcon-LLM} & 2 & \multicolumn{1}{l|}{DeepFloyd-IF-License} & 1 \\ \hline
  \multicolumn{1}{|l|}{CDLA-Permissive-2.0} & 2 & \multicolumn{1}{l|}{EPL-1.0} & 1 \\ \hline
  \multicolumn{1}{|l|}{CDLA-Permissive-1.0} & 2 & \multicolumn{1}{l|}{LPPL-1.3c} & 1 \\ \hline
  \end{tabular}
\end{table}

\clearpage

\begin{table*}[h]
  \caption{Summary of licensing details for ML projects with over 1K likes on Huggingface (Accessed on October 11, 2023). }
  \footnotesize
  \label{tab:MLP}
  \begin{tabular}{|p{2.1cm}|p{1.6cm}|p{2cm}|p{2.75cm}|p{3cm}|p{1.7cm}|p{2cm}|}
      \hline
      \rowcolor[gray]{.8}
      \textbf{ML Project} & \textbf{Task} & \textbf{Data License} & \textbf{Software License} & \textbf{Model License} & \textbf{Dataset} & \textbf{Risk Resource} \\ \hline
      
      Stable Diffusion v1-5 & Text to Image & CC-BY-4.0 & CreativeML-OpenRAIL-M & CreativeML-OpenRAIL-M & LAION-5B & Common Crawl \\ \hline
      
      BLOOM & Text Generation & \textit{Mixture} & \textit{Unknown} & BigScience-BLOOM-RAIL-1.0 & \textit{Crowdsourced} & Common Crawl, \newline Wikipedia, etc. \\ \hline

      OrangeMixs & Text to Image & \textit{Mixture} & \textit{Unknown} & CreativeML-OpenRAIL-M & \textit{Crowdsourced} & Danbooru \\ \hline

      ControlNet & Text to Image &  \textit{Unknown} & Apache-2.0 & OpenRAIL & \textit{Unknown} & n/a \\ \hline

      Openjourney & Text to Image &  CC-BY-NC-4.0 & \textit{Unknown} & CreativeML-OpenRAIL-M & Midjourney Gen & Midjourney Gen \\ \hline

      ChatGLM-6B & Text Generation &  \textit{Mixture} & Apache-2.0 & \textit{Custom} & the Pile, Wudao, \newline \textit{Crowdsourced} & PubMed,  Wikipedia, \newline arXiv, GitHub, etc. \\ \hline

      Llama2 & Text Generation &  \textit{Unknown} & Llama2 Community License & Llama2 Community License & \textit{Unknown} & n/a \\ \hline

      StarCoder & Text Generation &  \textit{Mixture} & Apache-2.0 & BigCode-OpenRAIL-M & The Stack & none \\ \hline

      Falcon-40B & Text Generation & ODC-By & Apache-2.0 & Apache-2.0 & RefinedWeb & Wikipedia, Reddit, \newline StackOverflow, etc. \\ \hline

      Waifu Diffusion & Text to Image & \textit{Mixture} & \textit{Unknown} & CreativeML-OpenRAIL-M & \textit{Unknown} & n/a \\ \hline

      Dolly-v2-12B & Text Generation & CC-BY-SA-3.0\&4.0 & MIT & MIT & databricks-dolly\newline-15k, the Pile & PubMed,  Wikipedia, \newline arXiv, GitHub, etc. \\ \hline

      Dreamlike Photoreal & Text to Image & \textit{Unknown} & \textit{Unknown} & \textit{Modified} CreativeML-\newline OpenRAIL-M & \textit{Unknown} & n/a \\ \hline

      Counterfeit & Text to Image & \textit{Unknow} & \textit{Unknown} & CreativeML-OpenRAIL-M & \textit{Unknown} & n/a \\ \hline

      GPT-2 & Text Generation & \textit{Mixture} & \textit{Modified} MIT & \textit{Modified} MIT & \textit{Crowdsourced} & WordPress, GitHub, \newline wikiHow, IMDb, etc. \\ \hline

      GPT-J-6B & Text Generation & \textit{Mixture} & Apache-2.0 & Apache-2.0 & the Pile & PubMed,  Wikipedia, \newline arXiv, GitHub, etc. \\ \hline

      LLaMA-7B & Text Generation & \textit{Mixture} & \textit{Custom} & \textit{Custom} & \textit{Crowdsourced} & GitHub, arXiv, etc. \\ \hline

      BERT & Fill Mask & \textit{Mixture} & Apache-2.0 & Apache-2.0 & Book Corpus, \newline Wikipedia (en) & Wikipedia (en) \\ \hline

      Whisper & ASR & \textit{Unknown} & MIT & MIT & \textit{Unknown} & n/a \\ \hline

      MPT & Text Generation & \textit{Mixture} & Apache-2.0 & Apache-2.0 & \textit{Crowdsourced} & Common Crawl, \newline Wikipedia, etc. \\ \hline

      Mistral-7B & Text Generation & \textit{Unknow} & Apache-2.0 & Apache-2.0 & \textit{Unknow} & n/a \\ \hline
  
  % <TOTAL>
  % Data License: All Permissive(2); Risk of Copyleft/Proprietary(11); Custom/Modified(0); Unknown(6);
  % Software License: Permissive(11); Risk of Copyleft/Proprietary(n/a); Custom/Modified(2); Unknown(6);
  % Model License: Permissive(15); Risk of Copyleft/Proprietary(n/a); Custom/Modified(4); Unknown(0);

  \end{tabular}
\end{table*}

\end{document}
\endinput
%%
%% End of file `sample-sigconf.tex'.
